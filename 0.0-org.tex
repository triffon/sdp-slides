\PassOptionsToClass{aspectratio=169}{beamer}
\documentclass{beamer}
\usepackage{sdp}

\title{Организация на курса}

\date{6 октомври 2022 г.}

\hypersetup{colorlinks,urlcolor=blue,linkcolor=white}

\begin{document}

\begin{frame}
  \titlepage
\end{frame}

\begin{frame}
  \frametitle{Екип}

  \begin{itemize}
  \item \textbf{1 група:} Никола Георгиев
  \item \textbf{2 група:} Ясен Алексиев{\tiny$^*$}
  \item \textbf{3 група:} Стефан Вартоломеев
  \item \textbf{4 група:} Свилен Андонов
  \end{itemize}
\end{frame}

\begin{frame}
  \frametitle{Схема за оценяване: СДП}

  \begin{itemize}
  \item Обобщено представяне на студентите от семестъра (ОПСС)
    \begin{itemize}
    \item Домашна работа: 1 бр. (Д)
    \item Проект: 1 бр. (П)
    \item Контролни работи: 2 бр. (K$_{1,2}$)
      \begin{itemize}
      \item Контролно 1: 24.11, 16:00
      \item Контролно 2: 5.01, 16:00
      \end{itemize}
    \item Бонус: обратна връзка от работа през семестъра ($0 \leq \text{ОВ} \leq 0{,}75$)\\
      \begin{equation*}
        \text{ОПСС} = \frac{\text{Д} + \text{П} + \text{K}_1 + \text{K}_2}4 + \text{ОВ}
      \end{equation*}
    \end{itemize}
  \item Изпити (И)
    \begin{itemize}
    \item Писмен изпит (ПИ)
    \item Теоретичен изпит (ТИ)\\[-7.5ex]
      \begin{equation*}
        \qquad\qquad\text{И} = \frac{\text{ПИ} + \text{ТИ}}2
      \end{equation*}
    \end{itemize}
  \end{itemize}
  \vspace{2ex}
  \begin{equation*}
    \text{Крайна оценка} = \left[ \frac{\text{ОПСС} + \text{И}}2 \right]
  \end{equation*}
\end{frame}

\begin{frame}
  \frametitle{Канали за комуникация}
  \begin{itemize}
  \item Learn.fmi: Бакалаври, зимен семестър 2022/2023
    \begin{itemize}
    \item КН
      \begin{itemize}
      \item \href{https://learn.fmi.uni-sofia.bg/course/view.php?id=8509}{СДП (КН) 2022/23}
      \end{itemize}
    \end{itemize}
  \item \href{https://teams.microsoft.com/l/team/19\%3aUZEWpZa64z9Ns2tWRqscadTdQgtEunExCO5p-zOo1Lg1\%40thread.tacv2/conversations?groupId=01c838dc-0d43-4b32-a71a-e209f21f0551&tenantId=9d05c5fb-e448-4700-8a58-e15b93c84ea9}{Екип в Microsoft Teams}: директно записване с код \tt{0q5dtbx}
  \item \href{https://teams.microsoft.com/l/meetup-join/19\%3a07f7f2f74bdf434cbcaefdca4d10427a\%40thread.tacv2/1664915548412?context=\%7b\%22Tid\%22\%3a\%229d05c5fb-e448-4700-8a58-e15b93c84ea9\%22\%2c\%22Oid\%22\%3a\%227354cd98-c695-434c-afd6-2d519f48fa53\%22\%7d}{Онлайн излъчване в Teams}
  \end{itemize}
\end{frame}

\begin{frame}
  \frametitle{План на курса}

  \begin{columns}[t,onlytextwidth]
    \column{0.5\textwidth}

    \begin{itemize}
    \item структури от данни
    \item алгоритми
    \item масив
    \item стек
    \item опашка
    \item свързан списък
      \begin{itemize}
      \item едносвързан
      \item двусвързан
      \item хетерогенен
      \end{itemize}
    \end{itemize}

    \column{0.5\textwidth}

    \begin{itemize}
    \item дърво
      \begin{itemize}
      \item двоично
      \item пирамида
      \item с произволен брой наследници
      \item двоично за търсене
      \item AVL дървета
      \item B-дървета
      \end{itemize}
    \item хеш таблица
    \item графи
    \item алгоритми за сортиране
    \end{itemize}

  \end{columns}
\end{frame}

\end{document}
