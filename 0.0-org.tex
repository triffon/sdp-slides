\documentclass[alsotrans,beameroptions={aspectratio=169}]{beamerswitch}
\usepackage{sdp}
\usepackage{qrcode}

\title{Организация на курса}

\date{3 октомври 2024 г.}

\hypersetup{colorlinks,urlcolor=blue,linkcolor=white}

\begin{document}

\begin{frame}
  \titlepage
\end{frame}

\begin{frame}
  \frametitle{Екип}

  \textbf{Семинари}
  \begin{itemize}
  \item \textbf{5 група:} Иван Арабаджийски
  \item \textbf{6 група:} Мартин Лаков
  \item \textbf{7 група:} Наделина Шипочка
  \item \textbf{8 група:} Алексис Датсерис
  \end{itemize}
  \vspace{1ex}
  \textbf{Практикум}
  \begin{itemize}
  \item слято за спец. Информатика и Компютърни науки, 1 и 2 поток
  \item разпределение в седем групи по избор
  \item започва от 7.10
  \end{itemize}
\end{frame}

\begin{frame}
  \frametitle{Схема за оценяване: СДП}

  \begin{itemize}
  \item Обобщено представяне на студентите от семестъра (ОПСС)
    \begin{itemize}
    \item Контролни работи: 2 бр. (K$_{1,2}$)
    \item Бонус: обратна връзка от работа през семестъра ($0 \leq \text{ОВ} \leq 0{,}75$)\\
      \begin{equation*}
        \text{ОПСС} = \frac{\text{K}_1 + \text{K}_2}4 + \text{ОВ}
      \end{equation*}
    \end{itemize}
  \item Изпити (И)
    \begin{itemize}
    \item Писмен изпит (ПИ)
    \item Теоретичен изпит (ТИ)\\[-7.5ex]
      \begin{equation*}
        \qquad\qquad\text{И} = \frac{\text{ПИ} + \text{ТИ}}2
      \end{equation*}
    \end{itemize}
  \end{itemize}
  \vspace{2ex}
  \begin{equation*}
    \text{Крайна оценка} = \left[ \frac{\text{ОПСС} + \text{И}}2 \right]
  \end{equation*}
\end{frame}

\begin{frame}
  \frametitle{Канали за комуникация}
  \begin{columns}[T,onlytextwidth]
    \begin{column}{.7\textwidth}
      \begin{itemize}
      \item Learn.fmi: Бакалаври, зимен семестър 2024/2025
        \begin{itemize}
        \item КН
          \begin{itemize}
          \item \href{https://learn.fmi.uni-sofia.bg/course/view.php?id=10454}{СДП (КН.2) 2024/25}
        \end{itemize}
      \end{itemize}
      \vspace{7ex}
    \item \href{https://discord.gg/fahBXftfpQ}{Discord сървър на ФМИ}
    \begin{itemize}
    \item код за покана \tt{fahBXftfpQ}
    \end{itemize}
  \end{itemize}              
\end{column}
\begin{column}{.3\textwidth}
  \qrcode{https://learn.fmi.uni-sofia.bg/course/view.php?id=10454}
  
  \vspace{4ex}
  \qrcode{https://discord.gg/fahBXftfpQ}
\end{column}
\end{columns}
\end{frame}

\begin{frame}
  \frametitle{План на курса}

  \begin{columns}[t,onlytextwidth]
    \column{0.5\textwidth}

    \begin{itemize}
    \item структури от данни
    \item масив (търсене и сортиране)
    \item стек
    \item опашка
    \item свързан списък
      \begin{itemize}
      \item едносвързан
      \item двусвързан
      \item хетерогенен
      \end{itemize}
    \end{itemize}

    \column{0.5\textwidth}

    \begin{itemize}
    \item дърво
      \begin{itemize}
      \item двоично
      \item пирамида
      \item с произволен брой наследници
      \item двоично за търсене
      \item AVL дървета
      \item B-дървета
      \end{itemize}
    \item хеш таблица
    \item графи
    \end{itemize}
  \end{columns}
\end{frame}

\end{document}
