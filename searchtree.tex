\documentclass{beamer}
\usepackage{sdp}

\title{Дървета за търсене}

\date{4 декември 2015 г.}

\titlegraphic{\includegraphics[height=0.35\textheight]{images/searchtree.png}}

\forestset{default/.style={baseline,for tree={fill=diagramblue,draw,circle,inner
      sep=0pt,minimum size=4ex,edge=->}}}

\forestset{scheme/.style=default,for tree={minimum size=6ex}}

\forestset{tri/.style={shape=isosceles triangle,shape border rotate=90,minimum height=6em,child anchor=north,anchor=north}}

\forestset{trismall/.style={tri,minimum height=4em}}

\forestset{remove/.style={tikz={\draw[thick,red] (!.north west) -- (!.south east); \draw[thick,red] (!.north east) -- (!.south west);}}}

\forestset{removesmall/.style={tikz={\draw[thick,red] (!.north west)++(1.5ex,1.5ex) -- ++(3ex,-3ex); \draw[thick,red] (!.north west)++(4.5ex,1.5ex) -- ++(-3ex,-3ex); \node at (!.north west) [xshift=3ex,circle,draw,inner sep=0.2ex]{Y};}}}

\newcommand{\samplebinordtree}{%
  \begin{forest}
    default [5 [3 [2 [1] [,phantom]] [4]] [8 [6 [,phantom] [7] ] [10]]]
  \end{forest}%
}

\begin{document}

\begin{frame}
  \titlepage
\end{frame}

\begin{frame}
  \frametitle{Дървета за търсене}
  \begin{itemize}
  \item Организация, която позволява бързо намиране на елементи в дървото
  \item Разчита на \textbf{линейна наредба} на елементите
  \item Основни операции:
    \begin{itemize}
    \item \tt{create()} --- създаване на празно дърво за търсене
    \item \tt{insert(x)} --- включване на елемент
    \item \tt{remove(x)} --- изключване на елемент
    \item \tt{search(x)} --- търсене на елемент
    \end{itemize}
  \item Обикновено елементите са двойки (ключ,стойност)
  \item Елементите са наредени относно ключовете си
  \item Стойностите носят данните на елемента
  \end{itemize}
\end{frame}

\section{Двоично дърво за търсене}

\begin{frame}
  \frametitle{Двоично дърво за търсене}
  \begin{definition}[Двоично дърво за търсене]
    \begin{itemize}
    \item Празното дърво $\bot$ е ДДТ
    \item $(X,L,R)$ е ДДТ, ако
      \begin{itemize}
      \item $X$ е по-голямо от от всички елементи в $L$
      \item $X$ е по-малко от от всички елементи в $R$
      \item $L$ и $R$ също са ДДТ
      \end{itemize}
    \end{itemize}
  \end{definition}
  \pause
  Пример:
  \vspace{-1em}
  \begin{center}
    \small
    \samplebinordtree
  \end{center}
\end{frame}

\begin{frame}
  \frametitle{Търсене на елемент}
  \begin{center}
    \begin{forest}
      scheme [{=X} [<X,tri] [>X,tri]]
    \end{forest}
  \end{center}
\end{frame}

\begin{frame}
  \frametitle{Включване на елемент}
  \small
  \begin{columns}[t,onlytextwidth]
    \begin{column}{0.3\textwidth}
      \begin{forest}
        scheme
        [X [Y<X,anchor=north] [,tri]]
      \end{forest}
    \end{column}
    \pause
    \begin{column}{0.3\textwidth}
      \begin{forest}
        scheme
        [X [,tri] [Y>X,anchor=north]]
      \end{forest}
    \end{column}
    \pause
    \begin{column}{0.4\textwidth}
      \begin{forest}
        scheme
        [X [<X,tri] [>X,tri]]
        \node at (!1.center) [yshift=3.5ex,xshift=-0.4ex,inner sep=0.2ex,draw,circle]{Y};
        \node at (!2.center) [yshift=4.1ex,xshift=0.8ex,inner sep=0.2ex,draw,circle]{Y};
      \end{forest}
    \end{column}    
  \end{columns}
\end{frame}

\begin{frame}
  \frametitle{Изключване на елемент}
  \small
  \begin{columns}[t,onlytextwidth]
    \begin{column}{0.3\textwidth}
      \begin{forest}
        scheme
        [\textbf{Y},remove [,tri,phantom] [,tri]]
      \end{forest}
    \end{column}
    \pause
    \begin{column}{0.3\textwidth}
      \begin{forest}
        scheme
        [\textbf{Y},remove [,tri] [,tri,phantom]]
      \end{forest}
    \end{column}
    \pause
    \begin{column}{0.4\textwidth}
      \begin{forest}
        scheme
        [X [<X,tri,removesmall] [>X,tri,removesmall]]
      \end{forest}
    \end{column}    
  \end{columns}
\end{frame}

\begin{frame}
  \frametitle{Изключване на елемент --- общ случай}
  \small
  \begin{columns}[t,onlytextwidth]
    \begin{column}{0.5\textwidth}
      \begin{forest}
        scheme
        [\textbf{Y},remove [<Y<M,tri] [>M,tri,parent anchor=south west [M [,trismall,phantom] [>M\\<M',align=center,trismall]] [,tri,phantom]]]
      \end{forest}
    \end{column}
    \pause
    \begin{column}{0.5\textwidth}
      \begin{forest}
        scheme
        [M [<M,tri] [>M,tri,parent anchor=south west [>M\\<M',align=center,trismall] [,trismall,phantom]]]
      \end{forest}
    \end{column}    
  \end{columns}
\end{frame}


\end{document}
