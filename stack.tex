\documentclass{beamer}
\usepackage{sdp}

\title{Стек}

\date{23 октомври 2015 г.}

\titlegraphic{\includegraphics[height=0.4\textheight]{images/stack.jpg}}

\begin{document}

\begin{frame}
  \titlepage
\end{frame}

\section{АТД стек}

\begin{frame}
  \frametitle{АТД: стек}

  Хомогенна линейна структура с организация ``последен влязъл --- пръв излязъл'' (LIFO)
  \vspace{1em}

  Операции
  \vspace{0.5em}
  \begin{itemize}
  \item \tt{create()} --- създаване на празен стек
  \item \tt{empty()} --- проверка за празнота на стек
  \item \tt{push(x)} --- включване на елемент на стек
  \item \tt{pop()} --- изключване на елемент от стек
  \item \tt{top()} --- последен елемент на стека
  \end{itemize}
\end{frame}

\begin{frame}
  \frametitle{АТД: стек}

  Свойства на операциите
  \vspace{0.5em}

  \begin{itemize}
  \item \tt{create().empty()} = \tt{true}
  \item \tt{s.push(x).empty()} = \tt{false}
  \item \tt{create().top()}, \tt{create().pop()} --- \alert{грешка}
  \item \tt{s.push(x).top()} = \tt x
  \item \tt{s.push(x).pop()} = \tt s
  \end{itemize}
\end{frame}

\begin{frame}
  \frametitle{Последователно представяне}
  \newcommand{\pha}{\phantom{$a_0$}}

  \begin{center}
    \begin{tabular}{|*{11}{c|}}
      \hline
      \rowcolor{diagramblue}
      $a_0$&$a_1$&$a_2$&\ldots&$a_n$&\alt<1>\pha{$a_{n+1}$}&\pha&\pha&\pha&\pha&\pha\\
      \hline
      \rowcolor{white}
      \multicolumn 4c{}&
      \multicolumn 1c{\onslide<1,3>{\bigg\uparrow}}&
      \multicolumn 1c{\onslide<2>{\bigg\uparrow}}&
      \multicolumn 5c{}\\
      \multicolumn 4c{}&
      \multicolumn 1c{\onslide<1,3>{top}}&
      \multicolumn 1c{\onslide<2>{top}}&
      \multicolumn 5c{}
    \end{tabular}
  \end{center}

  \begin{itemize}
    \item<2-> включване на елемент (push)
    \item<3-> изключване на елемент (pop)
  \end{itemize}
\end{frame}

\begin{frame}
  \frametitle{Свързано представяне}

  \begin{center}
    \scriptsize
    \begin{tabular}{cccc@{}c@{}cc}
      \onslide<2-4>{\nextcell{a_{n+1}}}&\nextcell{a_n}&\nextcell{a_{n-1}}&\hspace{1ex}\ldots&$\nextarrow$&\nextcell{a_1}&\nilcell{a_0}\\
      \onslide<2-4>{\bigg\uparrow}&\onslide<1,4-5>{\bigg\uparrow}&\multicolumn 5c{}\\
      \onslide<2-4>{\only<3-4>p\only<3>,\only<2-3>{top}}&\onslide<1,4-5>{top}&\multicolumn 5c{}
    \end{tabular}
  \end{center}

  \begin{itemize}
    \item<2-> включване на елемент (push)
    \item<3-> изключване на елемент (pop)
  \end{itemize}
\end{frame}

\section{Приложения на стек}

\subsection{Пресмятане на израз}

\begin{frame}
  \frametitle{Обратен полски запис}

  \begin{columns}[t,onlytextwidth]
    \column{0.6\textwidth}

    \begin{itemize}
    \item инфиксен запис:\\
      \tt{(1+2)*(3-4/5)}
    \item префиксен (полски) запис:\\
      \tt{*+12-3/45}
    \item постфиксен (обратен полски) запис\\
      \tt{12+345/-*}
    \end{itemize}

    \column{0.4\textwidth}
    
    \begin{center}
      \begin{forest} for tree={circle,draw,fill=diagramblue}
        [\tt* [\tt+ [\tt1] [\tt2]] [\tt- [\tt3] [\tt/ [\tt4] [\tt5]]]]
      \end{forest}
    \end{center}
  \end{columns}
\end{frame}

\begin{frame}
  \frametitle{Пресмятане на израз в обратен полски запис}

  \begin{center}
    \begin{tabular}{|p{20ex}p{25ex}@{}m{0pt}@{}}
      \hline
      \cellcolor{diagramblue}&\multicolumn 1{c|}{\cellcolor{diagramblue}обратен полски запис}&\\[5em]
      \cline{2-3}
      \multicolumn 1{|c|}{\cellcolor{diagramblue}резултати}&\multicolumn 1c{}&\\[5em]
      \cline{1-1}
    \end{tabular}
  \end{center}
\end{frame}


\begin{frame}
  \frametitle{Преобразуване на израз в обратен полски запис}

  \begin{center}
    \begin{tabular}{p{25ex}p{20ex}p{20ex}@{}m{0pt}@{}}
      \hline
      \multicolumn 1{|c}{\cellcolor{diagramblue}обратен полски запис}&\cellcolor{diagramblue}&\multicolumn 1{c|}{\cellcolor{diagramblue}инфиксен запис}&\\[5em]
      \cline{1-1} \cline{3-3}
      \multicolumn 1c{}&\multicolumn 1{|c|}{\cellcolor{diagramblue}операции}&\multicolumn 1c{}&\\[5em]
      \cline{2-2}
    \end{tabular}
  \end{center}
\end{frame}

\begin{frame}
  \frametitle{Директно пресмятане на израз}

  \begin{center}
    \begin{tabular}{|p{20ex}p{5ex}p{20ex}p{20ex}@{}m{0pt}@{}}
      \hline
      \cellcolor{diagramblue}&\cellcolor{diagramblue}&\cellcolor{diagramblue}&\multicolumn 1{c|}{\cellcolor{diagramblue}инфиксен запис}&\\[5em]
      \cline{2-2} \cline{4-4}
      \multicolumn 1{|c|}{\cellcolor{diagramblue}резултати}&\multicolumn 1c{}&\multicolumn 1{|c|}{\cellcolor{diagramblue}операции}&\multicolumn 1c{}&\\[5em]
      \cline{1-1} \cline{3-3}
    \end{tabular}
  \end{center}
\end{frame}

\subsection{Симулиране на рекурсия}

\begin{frame}
  \frametitle{Симулиране на рекурсия}

  \begin{itemize}
  \item Стекова рамка
    \begin{itemize}
    \item при извикване на функция
    \item при рекурсия
    \end{itemize}
  \item Стек вместо стекова рамка
  \item Пример: ход на коня
  \end{itemize}
\end{frame}

\end{document}
