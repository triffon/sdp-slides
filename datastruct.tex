\documentclass{beamer}
\usepackage{sdp}

\title[СД+А=П]{Структури от данни\\+\\алгоритми\\=\\програми}

\date{16 октомври 2015 г.}

\begin{document}

\begin{frame}
  \titlepage
\end{frame}

\section{Типове и структури}

\begin{frame}
  \frametitle{Тип данни}

  Инструмент за класификация на данните, характеризиращ се с:
  \begin{itemize}
  \item Множество от стойности
  \item Операции над стойностите
  \end{itemize}
  \vspace{1em}

  \pause

  \alert{За какво служат типовете?}

\end{frame}

\begin{frame}
  \frametitle{Типове данни (ТД)}

  Примитивни:
  \begin{itemize}
  \item булев (\tt{bool})
  \item целочислен (\tt{short}, \tt{int}, \tt{long}, \tt{unsigned})
  \item числа с плаваща запетая (\tt{float}, \tt{double})
  \item символен (\tt{char})
  \item указател (\tt*)
  \item псевдоним (\tt\&)
  \end{itemize}

  Съставни:
  \begin{itemize}
  \item масив (\tt{[]})
  \item структура / запис (\tt{struct})
  \item клас (\tt{class})
  \end{itemize}
\end{frame}

\begin{frame}
  \frametitle{Структури от данни (СД)}

  \begin{itemize}[<+->]
  \item СД са схеми за организация на даден вид данни с цел ефективност
  \item Всеки ТД в частност може да се разглежда като СД
  \item СД обикновено се реализират чрез потребителски дефинирани ТД
  \end{itemize}
\end{frame}

\begin{frame}
  \frametitle{Абстрактен тип данни (АТД)}

  \begin{itemize}[<+->]
  \item Формален модел на ТД или СД
    \begin{itemize}
    \item Множество от стойности
    \item Описание на имената и вида на операциите
    \item Описание на поведението и свойствата на операциите
    \end{itemize}
  \item Не налага конкретна организация
    \begin{itemize}
    \item (за разлика от СД)
    \end{itemize}
  \item Не налага конкретно представяне
    \begin{itemize}
    \item (за разлика от ТД)
    \end{itemize}
  \item Допуска една или повече реализации
  \item \alert{Какви са предимствата на АТД?}
  \end{itemize}
\end{frame}

\begin{frame}
  \frametitle{Видове описания на СД}

  Логическо описание
  \begin{itemize}
  \item състав
  \item компоненти
  \item операции
  \item свойства
  \end{itemize}
  \vspace{1em}

  Физическо описание
  \begin{itemize}
  \item организация на паметта
  \item представяне с един или повече ТД
  \item реализация на операциите
  \end{itemize}
\end{frame}

\begin{frame}
  \frametitle{Видове СД}

  \begin{itemize}
  \item Според вида на комопонентите си
    \begin{itemize}
    \item хомогенни
    \item хетерогенни
    \end{itemize}
  \item Според способността за промяна на размера
    \begin{itemize}
    \item статични
    \item динамични
    \end{itemize}
  \item Според вътрешната структура
    \begin{itemize}
    \item линейни (свързан списък)
    \item разклонени (дърво)
    \item мрежови (граф)
    \end{itemize}
  \end{itemize}
\end{frame}

\section{Алгоритми}

\begin{frame}
  \frametitle{Какво е алгоритъм?}

  \pause

  \textbf{Неформално:}

  Добре дефиниран набор от инструкции за извършване на дадено пресмятане.
  \vspace{2cm}

  \pause

  \textbf{Формално:}

  Машина на Тюринг
\end{frame}

\begin{frame}
  \frametitle{Масови задачи}

  \begin{itemize}[<+->]
  \item Масова задача: общ изчислителен проблем, който може да бъде формулиран за входни данни с произволен размер
    \begin{itemize}
    \item Пример: подреждане на елементи на масив във възходящ ред (сортиране)
    \end{itemize}
  \item Алгоритъмът като решение на масова задача
  \item Една масова задача може да има много възможни решения
  \item \alert{Как да сравнявамеме алгоритмите?}
  \item Добре е да имаме мярка за ефективността на алгоритъма
  \end{itemize}

  \pause
  \vspace{1em}
  Видове сложност
  \begin{itemize}
  \item времева сложност --- оценка на времето за изпълнение на алгоритъма
  \item пространствена сложност --- оценка на паметта използвана от алгоритъма
  \end{itemize}

\end{frame}

\begin{frame}
  \frametitle{Сложност на алгоритъм}

  \pause
  \vspace{1em}

  Как да оценим колко ресурс използва даден алгоритъм?
  \begin{itemize}[<+->]
  \item брой процесорни инструкции и брой байтове?
    \begin{itemize}
    \item не знаем на какъв процесор ще се изпълнява програмата!
    \end{itemize}
  \item брой ``атомарни операции'' и брой ``единици памет''?
    \begin{itemize}
    \item зависи колко големи данни подадем на алгоритъма
    \end{itemize}
  \item функция на броя операции или променливи в зависимост от големината на входа?
    \begin{itemize}
    \item точният брой може да варира в зависимост от конкретната реализация или език за програмиране
    \item но варирането \textbf{не трябва да зависи от големината на входа}
    \end{itemize}
  \end{itemize}
  \vspace{1em}

  \onslide<.->{
  \textbf{Асимптотична сложност}

  Оценка на функцията на сложност при произволно голям вход.
  }
\end{frame}

\begin{frame}
  \frametitle{$O$-нотация}

  \textbf{Проблем:} искаме да категоризираме функциите относно поведението им при големи стойности

  \textbf{Решение:} дефинираме класове от функции
  \vspace{1em}

  \pause

  Нека $f, g : \mathbb N \to \mathbb N$.
  \begin{eqnarray*}
    &f \in O(g) \Leftrightarrow \exists C \exists k \forall n \geq k (0 \leq f(n) \leq C\cdot g(n))& (\text{\small $f$ расте по-бавно от $g$})\\
    \pause
    &f \in \Omega(g) \Leftrightarrow \exists C \exists k \forall n \geq k (0 \leq C\cdot g(n) \leq f(n))& (\text{\small $f$ расте по-бързо от $g$})
  \end{eqnarray*}
  \pause
  Лесно се вижда, че $f\in O(g) \Leftrightarrow g\in \Omega(f)$.
  \pause

  \begin{equation*}
    f\in \theta(g) \Leftrightarrow \exists C_1 \exists C_2 \exists k \forall n \geq k (0 \leq C_1\cdot g(n) \leq f(n) \leq C_2\cdot g(n))
  \end{equation*}

  \pause

  Лесно се вижда, че $\Theta(g) = O(g) \cap \Omega(g)$.
\end{frame}

\begin{frame}
  \frametitle{$O$-нотация: примери}

  \begin{itemize}[<+->]
  \item $f \in O(f), f \in \Theta(f)$
  \item $1000n^2 \in O(n^3)$ (при $C = 1000$ е вярно, че $1000n^2 \leq 1000n^3$)
  \item $3n^2 + 100 \in \Theta(n^2)$, защото при $C_1 = 1, C_2 = 4, k = 10$ е вярно, че
    \begin{equation*}
      \forall n \geq 10 (0 \leq n^2 \leq 3n^2 + 100 \leq 4n^2)
    \end{equation*}
  \item $2^n \in O(3^n)$, $\log_2 (n) \in \Theta(log_{10}(n))$
    \begin{itemize}
    \item обикновено бележим просто $\log n$, понеже основата няма значение
    \end{itemize}
  \item $\log n \in O(n^{0.001})$, $n^{1000}\i O(1.001^n)$
  \item $\forall C (C \in \Theta(1))$, $O(1) = \Theta(1)$, $\Omega(1) = \mathbb N^{\mathbb N}$
  \end{itemize}
\end{frame}

\begin{frame}
  \frametitle{Видове сложност}

  \begin{itemize}[<+->]
  \item В най-лошия случай (песимистична)
    \begin{itemize}
    \item Какъв е максималният възможен брой операции (единици памет), които могат да са нужни на алгоритъма, за да реши задачата?
    \end{itemize}

  \item В най-добрия случай (оптимистична)
    \begin{itemize}
    \item Какъв е минималният възможен брой операции (единици памет), които може да извърши (използва) алгоритъмът, за да реши задачата?
    \end{itemize}

  \item В средния случай (средна)
    \begin{itemize}
    \item Ако считаме, че всеки възможен вход е равновероятен, какво е ``средното аритметично'' на броя операции (единици памет), които трябват при всички възможни входове?
    \end{itemize}

  \item В контекста на конкретна програма (амортизирана)
    \begin{itemize}
    \item Ако алгоритъмът ще се извиква няколко пъти в рамките на дадена програма, колко операции (единици памет) средно ще са му необходими за едно извикване?
    \end{itemize}
  \end{itemize}

\end{frame}

\begin{frame}[fragile]
  \frametitle{Пресмятане на сложност: пример}

\begin{verbatim}
for(int i = 0; i < n-1; i++)
  for(int j = n-2; j >= i; j--)
    if (a[j] > a[j+1]) {
      double x = a[j];
      a[j] = a[j+1];
      a[j+1]= x;
    }
\end{verbatim}

Да се оценят песимистична, оптимистична и средна времева сложност.
\end{frame}

\section{STL}

\begin{frame}
  \frametitle{Standard Template Library (STL)}

  Библиотека от шаблони, реализираща стандартни структури от данни и алгоритми.

  \begin{itemize}
  \item част от C++ Standard Library
  \item основни компоненти
    \begin{itemize}
    \item алгоритми (\tt{<algorithm>})
    \item контейнери
    \item функционални обекти (\tt{<functional>})
    \end{itemize}
  \item базирана на идеята за генерично програмиране
  \item ефективност: дава гаранции за сложност на алгоритми и операции над СД
  \end{itemize}
\end{frame}

\end{document}
