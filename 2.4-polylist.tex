\documentclass[alsotrans]{beamerswitch}
\usepackage{sdp}

\title{Хетерогенни контейнери}

\date{2 декември 2019 г.}

\titlegraphicx{\includegraphics[height=0.25\textheight]{images/polylist.jpg}\\
  \imageUntitledAttr{Clker-Free-Vector-Images}{https://pixabay.com/images/id-312107/}{Pixabay License}}

\forestset{
  classdiagram/.style={fill=diagramblue,baseline,draw,ellipse,edge={pointer},grow=90}
}

\begin{document}

\begin{frame}
  \titlepage
\end{frame}

\section{Дефиниция}

\begin{frame}
  \frametitle{Логическо описание}
  Контейнер, чиито елементи може да са от различен тип\\[2ex]
  Операции:
  \begin{itemize}
  \item всички операции на конкретната СД (напр. стек, опашка списък)
  \item изпълняване на еднотипна операция за всеки елемент
    \begin{itemize}
    \item \textbf{Пример:} извеждане
    \item реализацията на операцията може да е различна за всеки отделен тип
    \end{itemize}
  \item изпълняване на операция само за елементи от определен тип
    \begin{itemize}
    \item \textbf{Пример:} в хетерогенен списък от хора да се повиши оценката на всички студенти
    \end{itemize}
  \end{itemize}
\end{frame}

\begin{frame}
  \frametitle{Физическо представяне}
  \begin{center}
    \scriptsize
    \begin{tikzpicture}
      \doublecell{a1}{}
      \nextdoublecell{a2}{}{a1}
      \nextdoublecell{a3}{}{a2}
      \nextdots{a3}
      \dotsnextdoublecell{a4}{}
      \nextdoublecell{a5}{}{a4}
      \nullptr{a5next}

      \tikzset{
        classA/.style={rectangle, draw, fill=red},
        classB/.style={circle,    draw, fill=green},
        classC/.style={diamond,   draw, fill=yellow},
      }

      \node[classA,below=of a1data] (c1) {клас A};
      \node[classB,below=of a2data] (c2) {клас B};
      \node[classC,below=of a3data] (c3) {клас C};
      \node[classB,below=of a4data] (c4) {клас B};
      \node[classA,below=of a5data] (c5) {клас A};

      \foreach \i in {1,...,5} {
        \draw[pointer] (a\i data.center) to (c\i);
      }
    \end{tikzpicture}
  \end{center}
\end{frame}

\begin{frame}
  \frametitle{Реализация}
  Дефинира се общ базов клас (\tt{Object}) за елементите в списъка
  \begin{itemize}
  \item \tt{Object} обикновено е интерфейс
  \item декларира се чиста виртуална функция за всяка обща операция
  \end{itemize}
  \vspace{2ex}
  \begin{columns}[t,onlytextwidth]
    \begin{column}{0.7\textwidth}
      Дефинира се производен клас \tt{TypeObject} за всеки от тип \tt{Type}, от който искаме да сложим елементи в хетерогенния контейнер\\[2ex]
    \end{column}
    \begin{column}{0.3\textwidth}
      \ttfamily
      \begin{forest} for tree=classdiagram
        [TypeObject [Object] [Type]]
      \end{forest}
    \end{column}
  \end{columns}
  \vspace{2ex}
  \begin{itemize}
  \item \tt{TypeObject} множествено наследява \tt{Object} и \tt{Type}
  \item реализира се всяка от общите операции
  \item може директно да делегира към същата операция от \tt{Type}
  \end{itemize}
\end{frame}

\section{Примери}

\begin{frame}
  \frametitle{Списък от стекове и опашки}
  \textbf{Задача.} Да се реализира хетерогенен списък от стекове и опашки, който може да изпълнява операциите включване и изключване на елемент.\\[2ex]
  \pause
  \textbf{Решение:}\\[4ex]
  \begin{center}
    \ttfamily
    \begin{forest} for tree=classdiagram
      [,phantom [StackObject [Stack] [Object,name=o]] [QueueObject,name=qo [,phantom] [Queue]]]
      \draw[pointer] (qo) to (o);
    \end{forest}
  \end{center}
\end{frame}

\begin{frame}[fragile]
  \frametitle{Вложени списъци}
  \textbf{Задача.} Да се реализира хетерогенен списък, който може да съдържа числа или хетерогенни списъци от същия вид. Да се поддържат операциите извеждане (\tt{print}) и събиране на всички числа (\tt{flatten}).\\[2ex]
  \textbf{Пример:}
\begin{verbatim}
((1 (2)) (((3) 4) (5 (6)) () (7)) 8)
\end{verbatim}
  \pause
  \textbf{Решение:}\\[4ex]
  \begin{center}
    \ttfamily
    \begin{forest} for tree=classdiagram
      [,phantom [IntElement [,phantom] [Element,name=e]] [ListElement,name=le [,phantom] [LinkedList<Element*>]]]
      \draw[pointer] (le) to (e);
    \end{forest}
  \end{center}
\end{frame}

\end{document}
